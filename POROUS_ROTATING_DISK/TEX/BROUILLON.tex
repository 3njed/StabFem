\documentclass[article,11pt,a4paper,nomtoc]{LaRouviereClasse}
	\title{Axi Pourous Disk}
	\author{Adrien Rouviere}
	\date{\today}
	\hypersetup{pdftitle=Axi Pourous Disk,
				pdfauthor=Adrien Rouviere}

%\graphicspath{{Images/Brouillon/}}
	
\begin{document}
%ooooooooooooooooooooooooooooooooooooooooooooooooooooooooooooooooooooooooooooooooooooooooooooooooooooo

%%%%%%%%%%%%%%%%%%%%%%%%%%%%%%%%%%%%%%%%%%%%%%%%%%%%%%%%%%%%%%%%%%
\section{Équations physiques}

Dans le fluide $\Omega_f$ :
\begin{equation}
	\left(\vec{u}\cdot\gradd\right)\vec{u} = -\gradd p + \dfrac{2}{Re}\Divv\tens{D}(\vec{u}).
	\label{Fluide}
\end{equation}

Dans le disque $\Omega_p$ :
\begin{equation}
	\left(\dfrac{1}{\epsilon^2}\vec{u}\cdot\gradd\right)\vec{u} = -\gradd p + \dfrac{2}{\epsilon Re}\Divv\tens{D}(\vec{u}) - \dfrac{1}{Re}\dfrac{1}{Da}\left(\vec{u}-\vecg{\Omega}\wedge\vec{x}\right).
	\label{Poreux}
\end{equation}

Partout $\Omega$ :
\begin{equation}
	\Divv\vec{u} = \vecg{0}.
	\label{Div u}
\end{equation}

On note
\begin{equation}
	\mathcal{NS}(\vec{u},p) = -\mathcal{S}_1\left(\vec{u}\cdot\gradd\right)\vec{u} - \gradd p + \mathcal{S}_2\dfrac{2}{Re}\Divv\tens{D}(\vec{u}),
	\label{NS}
\end{equation}
où
\begin{equation}
\mathcal{S}_1 =
\begin{cases}
	1						& \text{dans }\Omega_f \\
	\frac{1}{\epsilon^2}	& \text{dans }\Omega_p
\end{cases}
\qquad\text{et}\qquad
\mathcal{S}_2 =
\begin{cases}
	1						& \text{dans }\Omega_f \\
	\frac{1}{\epsilon}		& \text{dans }\Omega_p
\end{cases}
\end{equation}

\section{Formulation faible}

\begin{equation}
	\underbrace{\iint_\Omega{q\cdot\eqref{Div u}\;\dd S}}_{(I)} + \underbrace{\iint_\Omega{\vec{v}\cdot\eqref{NS}}\;\dd S}_{(II)} + \underbrace{\iint_{\Omega_p}{\vec{v}\cdot\left[\dfrac{-1}{Re\ Da}\left(\vec{u}-\vecg{\Omega}\wedge\vec{x}\right)\right]\;\dd S}}_{(III)}.
\end{equation}

On pose
\begin{equation}
	\begin{bmatrix}
	\vec{u} \\ 
	p
	\end{bmatrix}
	=	\begin{bmatrix}
	\vec{u^\star} \\ 
	p^\star
	\end{bmatrix} + \begin{bmatrix}
	\vec{\delta u} \\ 
	\delta p
	\end{bmatrix}
\end{equation}

On obtient tout calculs faits :

\begin{multline}
	\iint_{\Omega}{q\cdot\Div\left(\vec{u^\star}\right)} + \iint_{\Omega}{q\cdot\Div\left(\vec{\delta u}\right)} + \iint_{\Omega}{-\mathcal{S}_1\;\dfrac{1}{2}\mathcal{C}\left(\vec{u^\star},\vec{u^\star}\right)\cdot\vec{q}} + \iint_{\Omega}{-\mathcal{S}_1\;\mathcal{C}\left(\vec{u^\star},\vec{\delta u}\right)\cdot\vec{q}} \\ + \iint_{\Omega}{p^\star\cdot\Div(\vec{v})} + \iint_{\Omega}{\delta p\cdot\Div(\vec{v})} + \iint_{\Omega}{-\mathcal{S}_2\;\tens{D}(\vec{v}):\tens{D}(\vec{u^\star})} + \iint_{\Omega}{-\mathcal{S}_2\;\tens{D}(\vec{v}):\tens{D}(\vec{\delta u})} \\ + \iint_{\Omega_p}{\dfrac{-1}{Re\;Da}\vec{v}\cdot\left(\vec{u}-\vec{u_s}\right)} + \iint_{\Omega_p}{\dfrac{-1}{Re\;Da}\vec{v}\cdot\vec{\delta u}} + \text{ C.L } = 0
\end{multline}

%ooooooooooooooooooooooooooooooooooooooooooooooooooooooooooooooooooooooooooooooooooooooooooooooooooooo
\end{document}

%%-----------------------------------------
%	\subsection{}