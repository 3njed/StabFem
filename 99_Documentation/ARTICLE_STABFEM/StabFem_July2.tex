\documentclass[10pt,blue,hyperref={pdfpagemode=FullScreen}]{beamer}
%\usepackage[latin1]{inputenc}
%\usepackage[T1]{fontenc}
\usepackage[frenchb]{babel}
\usepackage{aeguill}
\usepackage{graphics}
\usepackage{multimedia}
\usepackage{amsfonts}
\usepackage{amsmath}
\usepackage{float}
\usepackage{rotate}
\usepackage{epstopdf}


\usepackage[normalem]{ ulem }
\usepackage{soul}

%\usepackage{beamerthemePauline}
%\usepackage{beamerfontthemestructuresmallcapsserifPauline}
\usecolortheme{seahorse}
\setbeamertemplate{frametitle}
{
\begin{centering}
\insertframetitle
\par
\end{centering}
}

\title[]{ Rénion "StabFem" }
%\institute[]{  IMFT UMR CNRS 5502, Universit\'e Paul Sabatier Toulouse }
\author[D. Fabre]{David Fabre et al.}
\date{\vspace{1.cm} 02/07/2018}

\begin{document}


\setbeamertemplate{navigation symbols}{}
%\AtBeginSection{
%\frame{\tableofcontents[current]}
%}

\AtBeginSubsection[]
{
  \begin{frame}<beamer>
    \frametitle{Plan}
    \tableofcontents[currentsection,currentsubsection]
  \end{frame}
}
%\AtBeginSubsection{
%\frame{\tableofcontents[currentsection,currentsubsection]}
%}

%%%%%%%%%%%%%%%%%%%%%%%%%%%%%%%%%%%%%%%%%%%%%%%%
\frame{
\titlepage
}


%%%%%%%%%%%%%%%%%%%%%%%%%%%%%%%%%%%%%%%%%%%%%%%%
%\frame{
%\tableofcontents
%}


%\section{A simple "1D" problem }
%\subsection{Définition du problème}


%%%%%%%%%%%%%%%%%%%%%%%%%%%%%%%%%%%%%%%%%%%%%%%%
\begin{frame}{StabFem : Cahier des charges }
%Equations : 
%\begin{equation}
%$$\frac{\partial V }{\partial t} = \nu \frac{\partial^2 V}{\partial y^2} $$
%\label{eq:navier}
%\end{equation}
%(on a supposé pour simplifier qu'il n'y avait pas de gradient de pression autre qu'hydrostatique).
%Conditions limites : $\left. v\right|_{y=0} = 0$, $\left. v\right|_{y=h} = U(t)$  

\begin{itemize}[<+->]

\item StabFem : an open-source and easy-to-use software allowing a large range of computations in fluid mechanics.

\item Initially  oriented towards Global Stability Approches (Linear \& Nonlinear) but allowing a larger number of computations
(DNS, linear acoustics, etc...)

\item Easy to use/install/customize, 

\item Multi-platform (Unix, MacOs, Windows) and design to run on "light" computers (Laptops...)

\item Freeware, based on two softwares : FreeFem++ and \sl{Matlab} -> Octave ? Python ?

\item Developed as a collaborative project
(IMFT, Università di Salerno, ONERA, UPFL, ...)

\item Maintained on Github

\end{itemize}

\end{frame}



\end{document}

%\section{Objets mobiles}


