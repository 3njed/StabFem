

\documentclass[11pt,a4paper]{article}
 
\usepackage[francais]{babel}
\usepackage[utf8]{inputenc}
\usepackage[T1]{fontenc}
%\usepackage{fancyhdr}
\usepackage{graphicx, amsmath, amssymb, latexsym}
%\usepackage{natbib}
 
 
\setlength{\textheight}{26cm} % 24
\setlength{\topmargin}{-2cm}
\setlength{\textwidth}{17cm}  % 15
%\setlength{\oddsidemargin}{20mm} % marge gauche (im)paire = 1in. + x1mm
%\setlength{\evensidemargin}{0mm} % book : marge gauche paire = 1in. + x2mm
\setlength{\oddsidemargin}{-5mm} % marge gauche (im)paire = 1in. + x1mm
\setlength{\evensidemargin}{-5mm} % book : marge gauche paire = 1in. + x2mm
\setlength{\fboxsep}{5mm}

\newcommand{\e}  {\mathrm{e}}   
\newcommand{\dd}  {\textrm d}      
\newcommand{\im}  {\mathrm{i}} %\newcommand{\im}  {\textrm i}
\newcommand{\dpa}[2]  {\frac {\partial #1} {\partial #2}}   
\newcommand{\ddpa}[2] {\frac {\partial^{2} #1} {\partial #2 ^{2}}}   
\newcommand{\dto}[2]  {\frac {{\textrm d} #1} {{\textrm d} #2}}   
\newcommand{\ddto}[2] {\frac {{\textrm d}^{2} #1} {{\textrm d} #2 ^{2}}}
%\newcommand{\tensor}[1]{\stackrel{\Rightarrow}{#1}}

%\graphicspath{{../Figures/}{../FiguresPierre/}}
\graphicspath{Figures/}

%\makeindex

%------------------------------------------------------------------------------
\begin{document}

%\begin{titlepage}

\noindent
\textsc{Universit\'e Paul Sabatier \hfill Ann\'ee 2017/2018} \\
\textsc{M1 Mécanique \hfill majeure Ondes et Turbulence / Ondes}
%\vspace{1cm}

%\begin{center}

\vspace{.3cm}
%\section*{TP numérique : ondes sur une corde} 
\begin{center}
{\Huge \bf TP numérique} 

\vspace{.3cm}

{\Huge \bf Flûtes et pots d'échappements}
\end{center}

\vspace{.6cm}
\large

%\section{TP numérique : Tuyaux et pots d'échappement}

Dans ce TP on va résoudre à l'aide du logiciel FreeFem++ l'équation des ondes en régime harmonique, dans un domaine dont l'entrée est un tuyau de section 
$S_{in}$.

\vspace{.4cm}

\section{Partie théorique}

%Pour l'année prochaine :
 
% - densité 6 au lieu de 12
 
 %- tout encapsuler 
 
% - faire les impedances avant les plots
 
% - travailler sur Azteca (verifier instal)
 
 %- cas pot d'echappement : log-log 
 
 %- appeler les cas 1,2,3,4
 
 %- mettre Z0 = 1
 
%- virer l'option [1-2] pour le plot

%- verifier les options du plot (fill et isovalues)

\subsection{Rappel : impédance d'entrée d'un circuit acoustique}

On étudie un circuit acoustique (tuyau, pot d'échappement, silencieux ou autre) de section d'entrée $S_{in}$. Le circuit est forçé en régime harmonique à la pulsation $\omega$, c'est-à-dire qu'en 
entrée le flux acoustique (débit massique) $q$ et la pression acoustique $p'$ 
sont de la forme $q = U_{in}e^{-i \omega t}$ et $p' = p_{in} e^{-i \omega t}$. On définit alors l'impédance d'entrée du circuit par 
$$
Z_{in} = \frac{p_{in} }{U_{in}} . 
$$

Cette quantité (complexe) a les propriétés suivantes :

\begin{itemize}

\item Les minimums de $|Z_{in}|$ correspondent aux fréquences de résonance du circuit forcé en pression, 

\item Les maximums de $|Z_{in}|$ correspondent aux fréquences de résonance 
du circuit forcé en vitesse, 

\item La partie réelle (en général positive) est reliée au flux d'énergie injecté dans le circuit par le forçage en entrée, et la partie imaginaire est reliée au déphasage entre la pression et le débit.

\item L'impédance est directement reliée au coefficients de réflexion et de transmission (en intensité acoustique) par :
$$
R_i = \left|  \frac{ Z_{in} - Z_0}{Z_{in} + Z_0}  \right|^2 ; \qquad 
T_i = 1- R_i.
$$  

où $Z_0 = \rho c/S_{IN}$ est l'impédance caractéristique d'un tuyau de section égale à la section d'entrée $S_{IN}$ du circuit.

\end{itemize}

\subsection{Résolution numérique par éléments finis}

Si le circuit étudié admet une symétrie de révolution, on peut travailler en coordonnées axisymétriques et introduire un potentiel des vitesses de la forme  
$$
\Phi(r,z,t) = \phi(r,z) e^{-i \omega t}
$$

Le problème consiste alors à résoudre l'équation des ondes sous la forme 
\begin{equation}
\Delta \phi + k^2 \phi = 0  \qquad \mbox{ pour } (r,z) \in \Omega \mbox{ avec } k = \omega/c
\label{eq:Ondes}
\end{equation}
Couplé à des conditions limites adaptées sur les frontières du domaine (cf. TD). 

Le problème est résolu par une méthode d'éléments finis.

L'impédance du circuit est ensuite déduite du potentiel des vitesses par :
$$
Z_{in} = \frac{p_{in} }{U_{in}} =  \frac{ 1/S_{in} \int_{\Gamma_i} (i \rho \phi)  2 \pi r dr  } 
{ \int_{\Gamma_i} (\partial \phi / \partial z) 2 \pi r dr } 
$$

\subsection{Installation des programmes}


Pour ce TP on fournit deux types de programmes : 

- Un ensemble de programmes écrit avec le logiciel d'éléments finis FreeFem++ (fichiers de la forme \verb| ###.edp |). 

%Ces programmes sont à utiliser dans la fenêtre terminal par la commande suivante :

%\verb| > FreeFem++ ####.edp |

- Des 'drivers' matlab (fichiers de la forme \verb| SF_###.m |)  issus du projet collaboratif StabFem, permettant d'utiliser ces programmes dans l'environnement Matlab.

Outre Matlab, vous devrez donc installer et/ou télécharger les logiciels suivants :

- FreeFem++ (installé sur Azteca,  a télécharger gratuitement si vous souhaitez utiliser votre ordinateur personnel)

- La suite "StabFem" qui se télécharge en tapant dans une fenêtre terminal la commande suivante : 
\verb| git clone https://github.com/erbafdavid/StabFem |

{\small 
(  Cette commande devrait suffire sur un ordinateur linux ou mac, ou windows si vous disposez de l'utilitaire {\em git } ;  dans le cas contraire vous pouvez télécharger le fichier compacté disponible à l'adresse suivante : 
\verb| https://github.com/erbafdavid/StabFem/tree/master/ACOUSTICS_PIPES.tgz  |
)
}

Vous travaillerez par la suite dans le sous-répertoire \verb| ACOUSTICS_PIPES |.

Si vous n'arrivez pas à installer les drivers StabFem, vous pouvez toutefois utiliser les programmes FreeFem directement  en mode terminal (voir ci-dessous).
 
\subsection{Description et utilisation des programmes}


\begin{enumerate}

\item Les quatre premiers programmes, nommés 
\verb| Mesh_#.edp|
%Pipe_Flanged.edp |, 
%\verb| Mesh_Tube_58cm_Trou.edp |, 
 %\verb| Mesh_TailPipe.edp | et \verb|Mesh_ExpansionChamber.edp |
 servent à construire un maillage du domaine, dans les 4 cas étudiés.

\begin{itemize}
\item En mode terminal,  tapez la commande 

\verb| > FreeFem++ ####.edp |.

\item En mode Matlab/StabFem, le driver correspondant est le programme \verb| SF_Init |, a utiliser de la manière suivante :
 
\verb| >>  bf = SF_Init('Mesh_#.edp') |

Pour visualisez le maillage, vous pourrez utiliser la commande \verb| plotFF(bf,'mesh') |.
\end{itemize}


 
 \item Le programme \verb| FF_Acoustic.edp | calcule le potentiel des vitesses pour une valeur de $k$ donnée, en déduit la valeur de l'impédance $Z_{in}(k)$. 

\begin{itemize}
\item En mode terminal,  tapez la commande \verb| > FreeFem++ FF_Acoustic.edp |, puis entrez au clavier la valeur du nombre d'onde  $k$. Le programme affiche le potentiel des vitesses dans le domaine (appuyez sur entrée pour afficher le potentiel à des instants successifs). Le programme génère aussi un fichier de données \verb|Champs_P_U_Axe.txt|
contenant les valeurs de P et U sur l'axe de symétrie, ainsi qu'un fichier graphique 
\verb|Champ_Acoustique.eps| que vous pouvez utiliser par exemple pour illustrer un rapport...

\item En mode matlab/StabFem, le driver correspondant s'utilise de la manière suivante (par exemple pour $k=1$) :

\verb| >>  AC = SF_Acoustic(bf,'k',1) |

le résultat AC est une "structure matlab" dont les champs peuvent être exploités et tracés avec plotFF et plot (voir examples dans le script).

\end{itemize}
 
 \item Le programme \verb| FF_Acoustic_Impedance.edp | Calcule successivement les valeur de l'impédance $Z_{in}(k)$ dans un intervalle de $k$ compris entre $k_{min}$ et $k_{max}$, par pas $dk$. 
 
\begin{itemize}
\item En mode terminal,  tapez \verb| > FreeFem++ FF_Acoustic_Impedance.edp | puis entrez successivement au clavier $k_min$, $k_max$, $dk$. Les résultats sont écrits dans un fichier \verb| Impedances.txt | qui peut ensuite être utilisé par un autre programme pour tracer les courbes (Matlab ou autre).

\item En mode matlab/StabFem, le driver correspondant s'utilise de la manière suivante (par exemple pour l'intervalle $k \in [0,1]$) :

\verb| >>  IMP = SF_Acoustic_Impedance(bf,'k',[0:.02:1]) |

le résultat IMP est une "structure matlab" dont les champs IMP.k et IMP.Z sont des vecteurs de même dimension que l'on peut tracer l'un un fonction de l'autre (voir examples dans le script).
  
\end{itemize}

\end{enumerate}
  
%  \vspace{.3cm}
%Les programmes se lancent avec la commande suivante (a taper dans un terminal unix) :
%
%\verb| > FreeFem++ | {\em(nom du programme)}
%
%\vspace{.3cm}
%Les programmes peuvent également être lancés dans une fenêtre de commande ou un programme Matlab de la manière suivante :
%
%\verb| >> system('/usr/local/bin/FreeFem++| {\em(nom du programme)} \verb| ')|
%
%\vspace{.3cm}
%
% Outre ces programmes on fournit des scripts Matlab qui "encapsulent" les programmes FreeFem+ Afin de pouvoir les utiliser comme s'il s'agissait de programmes Matlab :
% 
%\begin{itemize}
%
%\item Les quatre scripts nommés \verb| Script_|{\em (nom du cas)}\verb|.m|
%utilisent le programme 
%
%\verb| Impedance_loop.edp | et permettent de tracer l'impédance $|Z(k)|$ et/ou le coefficient de transmission $T_i(k)$ pour chacun des cas.
% 
% \item Le script \verb| Plot_P_U_Axe.m | est basé sur le programme \verb| Impedance_plot.edp |
%et permet de tracer les courbes donnant la pression $P(0,z)$ et la vitesse $U(0,z)$ sur l'axe de symétrie du domaine (à partir des données lues dans le fichier \verb|Champs_P_U_Axe.txt|).
% 
% 
%\end{itemize} 
%  \clearpage
\section{  Cas étudiés }

\subsection{Tuyau cylindrique débouchant dans un demi-espace}

On considère tout d'abord un tuyau cylindrique de longueur $L=10$ et rayon $a = 1$, 
débouchant dans un demi-espace. Le problème est adimensionné en posant $\rho = c =1$ 
(on a alors directement $\omega = k$).

Si vous travaillez en mode matlab/StabFem, il vous suffira de faire tourner le programme suivant pour étudier ce cas :

\verb| SCRIPT_DEMO_ACOUSTIQUE.m |

{\small
(si vous préferez travailler en mode terminal, vous devrez taper sucessivement les commandes équivalentes dans la fenêtre terminal.)
}

\begin{enumerate}

\item Observez la structure du champ acoustique pour des basses fréquences ($k$ d'ordre 0.1) et des hautes fréquences ($k$ d'ordre 3). Qu'observe-t-on ?


\item A partir de la courbe de l'impédance calculée, identifiez les valeurs de $k$ correspondant aux deux premiers minimums de $Z_{in}$. Comparez aux prédictions théoriques de l'exercice 2.4.1.

\item A l'aide du programme \verb| FF_Acoustic.edp | et/ou du driver \verb| SF_Acoustic.m |,
étudiez la structure du champ acoustique pour les valeurs de $k$ précédemment identifiées. Que constate-t-on ?

\item Pour ces  valeurs de $k$, observez la courbe donnant la pression $P(x)$ sur l'axe de symétrie. Ou se trouvent les noeuds de pression ? Estimez la valeur de la "correction de longueur" $\Delta$ (exercice 2.2).

\item Mêmes questions pour les deux premiers maximums de $Z_{in}$. 


\item Tracez, sur la même figure, la valeur de $Z_{in}$ calculée par le logiciel FreeFem++ et celle prédite par la théorie (exercice 2.2).

\end{enumerate}

\subsection{Tuyau avec trou latéral}

Dans ce second cas on considère un tuyau de longueur $L = 58cm$ et de rayon $a =1cm$. 
On prend les caractéristiques de l'air : $\rho = 1.225 kg/m^3$ et $c = 34000 cm/s$. 
Le maillage correspondant est généré par le programme \verb| Mesh_2.edp |.

\begin{enumerate}
%\item
%Générez le maillage avec le programme \verb| Mesh_Tube_58cm_Trou.edp |.
\item
Observez la structure du champ acoustique pour différentes valeurs de $k$ à l'aide du programme \verb| FF_Acoustic.edp | et/ou du driver \verb| SF_Acoustic.m |.

\item 
Tracez $Z_{in}$ en fonction de la fréquence $f = \omega/2\pi$ à l'aide du programme 
\verb| FF_Acoustic_Impedance.edp | et/ou du driver \verb| SF_Acoustic_Impedance.m |.

\item Comparez avec la théorie (exercice 2.3). Quelles valeurs faut-il prendre pour les corrections de longueur $\Delta$ et $\Delta'$ ?

\item Quelle est la fréquence fondamentale de ce tuyau dans le cas d'un forçage en pression (flûte) et d'un forçage en vitesse (clarinette) ? A quelles notes ces fréquences correspondent-elles ? 

\end{enumerate}

\subsection{Pot d'échappement}

On considère maintenant le cas d'un pot d'échappement (exercice 2.3.3).
Celui-ci est modélisé par un tuyau de rayon $a$, relié au travers d'une
ouverture d'aire $A$ à un résonateur de Helmholtz de volume $V$. 

Le maillage fourni correspond aux valeurs $a=1,A=6.28,V=388.8$.

(le programme ne permettant de traiter que des géométrie axisymétriques, l'ouverture est supposée de forme annulaire et de longueur $h$, l'aire vaut donc $A = 2\pi a h$)

Le maillage est généré avec le programme \verb| Mesh_3.edp |.


\begin{enumerate}
%\item
%Générez 
%\item
%Observez la structure du potentiel des vitesses pour différentes valeurs de $k$ à l'aide du programme \verb| Impedance_plot.edp |.
\item 
Tracez $Z_{in}$ (figure 1) ainsi que $T_i$ (figure 4 (échelle linéaire) et 5 (échelle logarithmique))  à l'aide du driver Mablab 
\verb| SF_Acoustic_Impedance.m |., pour $k$ dans l'intervalle $[0-0.02]$.

\item Une modélisation simplifiée (exercice 2.3.3) prédit que le coefficient de transmission vaut 
$$
T_i  = \frac{1}{1+(k/k_c)^2}, \mbox{ avec } k_c = \frac{2 A}{V}.
$$
Comparez cette prédiction avec le résultat du programme.

\item Pour des valeurs de $k$ plus grandes, on observe des maximums et minimums de $|Z|$ pour des valeurs précises. Illustrez la structure du champ acoustique dans ces cas (dans le domaine et sur l'axe). Peut-on prédire les valeurs de $k$ correspondantes (cf. exercice 3.1...)


%Ce modèle est également tracé par le script sur les figures 4 et 5. Que constate-t-on ?

%\item 
%Modifiez le script \verb| Script_TailPipe.m |, pour tracer le coefficient de transmission pour $k$ dans l'intervalle $[0-1]$. Que constate-t-on ? Expliquez.

%On pourra utiliser le programme \verb| Impedance_plot.edp | et/ou le script 

%\verb| Plot_P_U_Axe.m | pour observer la structure du champ acoustique pour certaines valeurs de $k$.  

\end{enumerate}



\subsection{Chambre d'expansion}

On considère finalement le cas d'une chambre d'expansion (exercice 2.3.2).

Le maillage correspondant est généré avec le programme \verb| Mesh_4.edp |.


\begin{enumerate}
%\item
%Générez le maillage avec le programme \verb| Mesh_ExpansionChamber.edp |.
\item
Observez la structure du champ acoustique pour différentes valeurs de $k$ à l'aide du programme \verb| FF_Acoustic.edp | et/ou du driver \verb| SF_Acoustic.m |.

\item 
Tracez $Z_{in}$ en fonction de $k$ ainsi que $T$ en fonction de $k$ à l'aide du script Mablab 
\verb| SF_Acoustic_Impedance.m |.

\item 
Comparez avec les prédictions théoriques (exercices 2.1.2 et 2.3.2).

\end{enumerate}


\subsection{Autres cas (bonus...)}

En explorant les programmes fournis vous trouverez 5 autres programmes construisant des maillages pour des géométries différentes. N'hésitez pas à jouer avec !


\section{Travail demandé}

Parmi les cas proposés, vous choisirez deux cas d'étude.

Pour chacun des cas choisis :
\begin{itemize}
\item Illustrez à l'aide des résultats du programme le comportement acoustique du circuit au travers d'un choix judicieux de courbes générées par les programmes, qui seront commentés de manière physique en lien avec l'application considérée (instrument de musique ou atténuateur de bruit).

\item Comparez avec les éléments de théorie vus en cours et/ou en TD.

\item Commentez et expliquez les ressemblances et éventuelles différences.
\end{itemize}

On précise que seules les figures disposant d'une légende et commentées dans le rapport seront prises en compte dans l'évaluation.




\end{document}



\subsection{Application 1 : instruments de musique}

Dans ce cas on s'intéresse aux maximums et minimums de l'impédance $|Z|$.

\subsubsection{Tuyau cylindrique}

Théorie : 

$$
Z_{IN} = Z_0 \frac{Z_L \cos k L + j Z_0 \sin k L}{j Z_L \sin k L + Z_0 \cos k L}
$$

avec 

$$
Z_L = Z_0 \left( \frac{k^2 a^2}{4} + j k \Delta \right) 
$$  


\subsubsection{Tube cylindrique avec trou latéral}

Théorie : formule précédente en remplaçant $L$ par $L-D$ et $\Delta$ par $\Delta'$.

\subsubsection{Tube conique}

Cone tronqué.

Théorie :

$$
Z_{IN} =...
$$


(Rossing, page 211).

\subsubsection{Pavillon exponentiel}

\subsection{Silencieux et pot d'échappement}

Dans ce cas on s'intéresse au coefficient de transmission.

\subsubsection{Silencieux simple constitué d'une chambre d'expansion}

Théorie : 

\subsubsection{Pot d'échappement}

Théorie :

\end{document}






